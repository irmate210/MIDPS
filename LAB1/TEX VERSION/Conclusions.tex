\section*{Concluzie}
\phantomsection

Am instalat Borland C++ Builder 6. Prima problema a fost un error straniu care a fost rezolvat peste vreo jumate de ora de forumuri, cu Open as Administrator. Am deschis Google si Youtube si am inceput sa ma uit macar Basic tutorials la acest Builder. Dupa ceva timp am inteles principiul de lucru a acestui soft.\\
Am inceput lucrul cu cele mai simple componente Label, Button, Edit si altele. Am inceput sa construiesc cel mai simplu lucru posibil, Panel1 - Visible=false. Dupa ceva timp am inceput sa caut cum se fac lucruri putin mai serioase. Am creat cu ajutorul a 2 componente Edit si un Button, un calculator care aduna 2 numere. Am creat cu ajutorul a unor RadioBox-uri optiuni pentru a schimba stilul unui text, culoare textului si marimea lui. Am folosit de asemenea si un Combox pentru a schimba fontul. Dupa aceasta am decis ca sunt gata pentru primul laborator. Am inceput cu crearea Timer-ului. Cu ajutorul a 2 Button-uri si unui Label am facut ca timerul sa se porneasca, opreasca si sa se faca reset. De asemenea am inteles cum se face functia ON/OFF pe aceeasi tasta(conditia if trebuie sa lucreze, if=true then false, else true). Am inceput crearea unor grafice cu ajutorul PaintBox. Cu ajutorul lui Canvas am desenat un BarGraf si o Diagrama. Aceeasi idee este la obiectul Grafica pe Calculator si de asta nu mi-a fost greu sa lucrez aici. Dupa, am inceput sa creez incrementarea si decrementarea cu ajutorul a 2 butoane. Nu era chiar asa de tirziu cind am terminat laboratorul, dar mi-a fost interesant ce se mai poate interesant de facut. Am inceput sa ma uit cum se poate de inclus o fotografie. Peste 10 minute, fotografia a fost inclusa in program cu ajutorul componentei Image. Am facut optiunea de a inchide si de a deschide fotografia.\\
In concluzie pot spune ca cu ajutorul C++ Builder putem crea diferite interfate pentru diferite sarcine.

\clearpage